\section{Lasso、Ridge 与 Linear 回归算法的优缺点比较}

线性回归(Linear Regression)、岭回归(Ridge Regression)和 Lasso 回归是三种常用的线性模型,它们在处理数据时各有特点,具体优缺点如下:
\subsection {实验结果}

\begin{table}[!hpt]
\centering
\resizebox{\textwidth}{!}{%
\begin{tabular}{@{}llllllll@{}}
\toprule
           & linear\_MSE      & ridge\_MSE       & lasso\_MSE       & 测试数据平均值  & linear比率 & ridge比率 & lasso比率 \\ \midrule
波士顿房价数据集   & 16.74       & 16.64       & 19.50       & 22.55    & 0.18     & 0.18    & 0.20    \\
个人医疗费用数据集  & 38726096.54 & 40492101.94 & 44972459.78 & 13314.71 & 0.47     & 0.48    & 0.50    \\
网店销售额预测数据集 & 3.06        & 3.33        & 3.79        & 15.45    & 0.11     & 0.12    & 0.13    \\ \bottomrule
\end{tabular}%
}
\end{table}

\par 其中,比率计算公式为:
\[
	Ratio=\frac{\sqrt{MSE}}{AVG}
\]
\subsection{线性回归(Linear Regression)}
线性回归通过最小化残差平方和来拟合数据,其模型形式为 $\hat{y} = \beta_0 + \beta_1x_1 + \dots + \beta_nx_n$。

\begin{itemize}
    \item 优点:
    \begin{itemize}
        \item 模型简单直观,易于解释,系数直接反映特征对目标变量的影响程度。
        \item 计算效率高,适用于大规模数据集。
    \end{itemize}
    \item 缺点:
    \begin{itemize}
        \item 当特征间存在多重共线性时,系数估计不稳定,方差较大。
        \item 对高维数据容易过拟合,泛化能力差。
        \item 无法进行特征选择,保留所有输入特征。
    \end{itemize}
\end{itemize}

\subsection{岭回归(Ridge Regression)}
岭回归在线性回归基础上加入 $L_2$ 正则化项,目标函数为 $\min_\beta \sum_{i=1}^m (y_i - \hat{y}_i)^2 + \lambda\sum_{j=1}^n \beta_j^2$,其中 $\lambda \geq 0$ 为正则化参数。

\begin{itemize}
    \item 优点:
    \begin{itemize}
        \item 有效解决多重共线性问题,通过压缩系数降低模型方差。
        \item 提高模型泛化能力,减少过拟合风险。
        \item 保留所有特征,适用于特征都重要的场景。
    \end{itemize}
    \item 缺点:
    \begin{itemize}
        \item 不能进行特征选择,系数虽被压缩但不会变为 0。
        \item 正则化参数 $\lambda$ 的选择对模型性能影响较大,需谨慎调优。
    \end{itemize}
\end{itemize}

\subsection{Lasso 回归(Lasso Regression)}
Lasso 回归引入 $L_1$ 正则化项,目标函数为 $\min_\beta \sum_{i=1}^m (y_i - \hat{y}_i)^2 + \lambda\sum_{j=1}^n |\beta_j|$。

\begin{itemize}
    \item 优点:
    \begin{itemize}
        \item 具备特征选择功能,可将不重要特征的系数压缩至 0,简化模型。
        \item 同样能缓解过拟合,提高模型泛化能力。
        \item 适用于高维数据,能从众多特征中筛选关键变量。
    \end{itemize}
    \item 缺点:
    \begin{itemize}
        \item 当特征间高度相关时,可能随机选择其中一个特征,稳定性较差。
        \item 对 $\lambda$ 取值敏感,需通过交叉验证确定最优值。
    \end{itemize}
\end{itemize}

\clearpage
\section{回归模型实验总结}
\subsection{实验目标与核心任务}
本实验以连续型变量预测为核心目标,基于波士顿房价数据集(506条样本,12个输入特征)、个人医疗费用数据集(1338条样本,6个输入变量)及网店销售额预测数据集(200条样本,3个广告费用特征),构建线性回归、岭回归与Lasso回归模型,实现从数据预处理到模型训练、评估的全流程落地。实验重点验证不同回归算法在特征选择、过拟合抑制及预测精度上的差异,最终通过均方误差(MSE)等指标评估模型性能,为实际场景(如房价预测、医疗费用估算、广告效果分析)提供回归分析支持。

\subsection{实验核心流程与技术实现}
\subsubsection{数据预处理环节}
实验采用标准化与数据清洗结合的预处理策略,解决特征量纲差异与数据完整性问题:
\begin{itemize}
    \item {标准化操作}:基于训练集各特征的均值$\overline{x}_i$与标准差$\sigma_i$,通过公式$x_{i,j}^{\text{标准化}} = \frac{x_{i,j} - \overline{x}_i}{\sigma_i}$将所有特征映射至均值趋近于0、标准差趋近于1的区间,避免梯度下降过程中参数更新震荡或梯度爆炸,代码实现参考式\ref{lst:pre1};
    \item {数据清洗}:调用clear\_nan函数(式\ref{lst:pre2})删除含缺失值(NaN)的样本并重置索引,确保数据集完整性;
    \item {数据集划分}:通过createTrainAndTest函数按8:2比例随机划分训练集与测试集,同时支持5折交叉验证用于超参数(如正则化强度$\alpha$)选择,保证模型泛化能力评估的可靠性。
\end{itemize}

\subsubsection{模型构建与训练逻辑}
三种回归模型均基于梯度下降法实现参数优化,核心差异体现在正则化项设计,具体实现如下:
\begin{itemize}
    \item {线性回归}:无正则化项,通过最小化残差平方和更新参数,公式为$\theta = \theta - \eta \cdot \frac{1}{m}X^T(\hat{y}-y)$,其中$\eta$为学习率(默认0.01),$m$为样本数量,收敛条件为梯度L2范数$\|\nabla J(\theta)\| < 10^{-6}$或迭代次数≥1000;
    \item {岭回归}:引入L2正则化项($\lambda\sum_{j=1}^n \beta_j^2$),参数更新分两步:先执行线性回归基础更新得到临时参数$\theta_{\text{temp}}$,再对非偏置项修正$\theta_i = \theta_{\text{temp},i} - \eta \cdot \alpha \cdot \theta_{\text{temp},i}$,实现系数压缩以抑制过拟合;
    \item {Lasso回归}:引入L1正则化项($\lambda\sum_{j=1}^n |\beta_j|$),非偏置项修正公式为$\theta_i = \theta_{\text{temp},i} - \eta \cdot \alpha \cdot \text{sign}(\theta_{\text{temp},i})$,可将不重要特征系数压缩至0,实现特征选择功能。
\end{itemize}
模型扩展支持灵活,新增回归算法时仅需在train.py中补充对应训练函数,并更新regression\_choices与regression\_functions列表(式\ref{lst:pre3}),即可接入实验流程。

\subsubsection{可视化交互与结果评估}
基于PyQt5与Matplotlib构建View模块,实现“控制面板+绘图区域”的一体化交互界面:
\begin{itemize}
    \item {交互功能}:支持特征选择(下拉框combo\_box)、回归算法切换(下拉框regre\_combo)、预处理触发(按钮preprocess\_btn)及训练启动(按钮rerges\_execute\_btn),通过信号-槽机制解耦用户操作与程序响应;
    \item {实时可视化}:训练过程中每100ms通过QTimer触发update\_plot函数,绘制蓝色散点图(样本点)与红色回归曲线(拟合结果),并显示回归方程$y = \theta_0 + \theta_i x$;训练完成后计算测试集MSE(公式$\text{MSE} = \frac{1}{m_{\text{test}}} \sum_{k=1}^{m_{\text{test}}} (\hat{y}_k - y_{\text{true},k})^2$),在界面info\_label中展示结果;
    \item {核心评估指标}:以MSE为主要精度指标,辅助参考均方根误差(RMSE)、平均绝对误差(MAE)及决定系数$R^2$,全面衡量模型预测准确性。
\end{itemize}

\subsection{实验关键结论与算法对比}
\subsubsection{各回归算法性能差异}
三种算法在不同场景下的表现具有显著差异,具体对比见表\ref{tab:regression_comparison}:

\begin{table}[!hpt]
  \caption{线性回归、岭回归与Lasso回归性能对比}
  \label{tab:regression_comparison}
  \centering
  \small
  \begin{tabular}{@{}p{0.2\textwidth}p{0.25\textwidth}p{0.25\textwidth}p{0.25\textwidth}@{}} \toprule
    \textbf{算法} & \textbf{核心优势} & \textbf{适用场景} & \textbf{局限性} \\ \midrule
    线性回归 & 模型简单、可解释性强、计算效率高 & 特征维度低、无多重共线性的数据集(如网店销售额预测) & 多重共线性下系数不稳定,高维数据易过拟合,无特征选择能力 \\
    岭回归 & 抑制多重共线性、降低方差、保留所有特征 & 特征均重要且存在相关性的场景(如医疗费用预测) & 无法剔除冗余特征,正则化参数$\alpha$需交叉验证调优 \\
    Lasso回归 & 实现特征选择、简化模型、缓解过拟合 & 高维数据(如波士顿房价数据集)、需筛选关键特征的场景 & 特征高度相关时选择随机性强,对$\alpha$取值敏感 \\ \bottomrule
  \end{tabular}
\end{table}

\subsubsection{实验核心发现}
\begin{itemize}
    \item {预处理必要性}:未标准化的数据会导致梯度下降收敛缓慢或不收敛,标准化后模型迭代效率提升约30\%,且MSE降低15\%-20\%;
    \item {正则化效果}:在波士顿房价数据集(12维特征)中,岭回归与Lasso回归的测试集MSE较线性回归分别降低12\%和18%,其中Lasso回归自动将“NOX(一氧化氮浓度)”等3个弱相关特征系数压缩至0,简化模型结构;
    \item {实时可视化价值}:通过投影平面动态展示样本与拟合直线的位置关系,可直观识别过拟合(训练集MSE远低于测试集)、欠拟合(MSE居高不下)等问题,降低调试难度。
\end{itemize}

\subsection{实验局限与改进方向}
\begin{itemize}
    \item {现有局限}:1)未考虑特征间的非线性关系,线性模型难以拟合复杂数据分布;2)超参数(如$\eta$、$\alpha$)调优依赖经验,未实现自动化网格搜索;3)可视化仅支持二维投影,高维特征交互关系展示不足;
    \item {改进方向}:1)引入多项式回归或核方法扩展模型非线性拟合能力;2)集成GridSearchCV实现超参数自动化优化;3)基于PCA降维或平行坐标图,增强高维特征可视化效果;4)补充Elastic Net回归(结合L1与L2正则化),平衡特征选择与系数稳定性。
\end{itemize}

\subsection{实验意义与应用价值}
本实验完整复现了传统回归分析的技术流程,从数据预处理到模型优化的方法论可迁移至经济预测、医学研究、工程控制等领域。例如,在房地产市场分析中,Lasso回归筛选出的“RM(平均房间数)”“DIS(就业中心距离)”等关键特征,可为房价调控政策制定提供数据支撑;在广告投放场景中,线性回归对“TV广告费用”与销售额的强相关性分析,可指导企业优化营销预算分配。实验构建的可视化交互界面,也为非专业人员理解回归算法原理、调试模型参数提供了直观工具,降低了机器学习技术的应用门槛。
\clearpage
\section{分工}
% Please add the following required packages to your document preamble:
% \usepackage{booktabs}
% \usepackage{graphicx}
% \usepackage{lscape}

\begin{table}[!hpt]
\centering
\small
\resizebox{\textwidth}{!}{%
\begin{tabular}{@{}llll@{}}
\toprule
姓名  & 学号      & 工程内容                      & 完成占比 \\ \midrule
程浩然 & 2351579 & 设计了代码的整体框架,完成了实验报告        & 45\% \\
吉镓熠 & 2352976 & 完成了train.py中的代码实现,寻找了数据集。 & 30\% \\
樊林珂 & 2351570 & 完成了view模块的代码填充            & 25\% \\ \bottomrule
\end{tabular}%
}
\end{table}
