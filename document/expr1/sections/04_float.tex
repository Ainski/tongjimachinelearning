\section{可视化交互界面(View模块)设计与实现}
View模块作为实验系统的用户交互核心,基于PyQt5与Matplotlib构建了兼具数据可视化与模型控制功能的图形界面。该模块通过面向对象设计封装了界面布局、交互逻辑与实时绘图功能,实现了特征选择、回归算法切换、数据预处理、模型训练及结果可视化的一体化操作。其核心实现机制如下:

\subsection{类结构与核心属性设计}
\par 界面功能通过ScatterPlotWindow类(继承自PyQt5的QMainWindow)实现,该类封装了界面元素、数据状态与交互方法,关键属性包括:
\par 数据存储:x\_data(特征数据,DataFrame)、y\_data(目标变量,Series)、x\_test/y\_test(测试集数据)及预处理参数(x\_means/x\_stds/y\_means/y\_stds);
\par 界面状态:当前选中特征(current\_feature)、当前回归算法(current\_regre)、模型参数(theta)及训练状态标识(show\_regre\_result);
\par  可视化组件:Matplotlib的Figure与FigureCanvas用于绘图,PyQt5的QComboBox/QPushButton/QLabel构成控制面板。

类的初始化方法(\_\_init\_\_)通过接收特征数据、目标变量、特征名称列表等参数完成初始状态设置,并调用init\_ui()构建界面、connect\_signals()绑定交互逻辑、update\_plot()绘制初始散点图。

\subsection{界面布局与控件组织}
界面采用“控制面板+绘图区域”的垂直布局(QVBoxLayout),各组件层次分明且功能独立:
\par  \textbf{信息提示区}:顶部QLabel(info\_label)用于显示系统状态(如“数据预处理完成”“训练失败:xxx”);
\par  \textbf{控制交互区}:水平布局(QHBoxLayout)包含4个核心控件:
\begin{itemize}
  \item 特征选择下拉框(combo\_box):加载feature\_names列表,支持用户切换X轴显示的特征;
  \item  预处理按钮(preprocess\_btn):点击触发run\_preprocess()方法,执行数据标准化并更新绘图;
  \item 回归算法下拉框(regre\_combo):加载regression\_choices列表(如线性回归、岭回归等),支持算法切换;
  \item  训练按钮(rerges\_execute\_btn):点击触发run\_regre()方法,启动模型训练流程;
\end{itemize}

\par  \textbf{绘图显示区}:占界面主体的FigureCanvas,用于渲染散点图与回归曲线,通过setSizePolicy设置为可扩展布局,适配窗口大小变化。

\subsection{信号与槽机制:交互逻辑实现}
模块基于PyQt5的信号-槽(Signal-Slot)机制实现用户操作与程序响应的解耦,核心信号与绑定关系如下:
\par  特征选择信号(feature\_changed):当combo\_box选择变化时,通过on\_feature\_changed方法发射信号,触发on\_feature\_update更新当前特征并调用update\_plot()重绘;
\par 算法选择信号(regre\_changed):当regre\_combo选择变化时,通过on\_regre\_changed方法发射信号,触发on\_regre\_update更新当前算法;
\par  按钮点击事件:预处理按钮绑定run\_preprocess(),训练按钮绑定run\_regre(),实现对应功能的触发。

这种设计使界面交互逻辑清晰可扩展,新增控件时只需添加相应信号与槽函数即可。

\subsection{核心功能实现流程}

\subsubsection{数据预处理与状态更新}
点击“执行数据预处理”按钮后,run\_preprocess()方法调用preprocess\_data函数对x\_data和y\_data执行标准化(基于均值与标准差),并将处理后的数据、统计参数(x\_means等)保存为类属性。预处理完成后,通过update\_plot()刷新散点图,并在info\_label显示状态提示;若失败则捕获异常并显示错误信息。

\subsubsection{模型训练与实时可视化}
点击“执行回归”按钮后,run\_regre()方法初始化模型参数(theta)、迭代计数器(current\_iter)与收敛标识(done),并构建含偏置项的特征矩阵X。随后启动QTimer定时器,每100ms触发一次train\_step()方法,实现以下流程:
\par 调用当前选中的回归算法(通过regression\_functions映射获取对应函数)执行参数更新;
\par  更新迭代次数与收敛状态,并在info\_label显示实时训练信息;
\par  调用update\_plot()重绘界面,此时show\_regre\_result为True,触发回归曲线绘制;
\par  当满足收敛条件(done=True)时,停止定时器,计算并显示测试集MSE(调用calculate\_mse)。

\subsubsection{动态绘图逻辑}
update\_plot()方法是可视化核心,负责根据当前状态绘制散点图与回归曲线:
\par 清除当前绘图区(self.figure.clear()),创建子图并绘制特征与目标变量的散点图(蓝色点,带黑色边缘);
\par 若训练完成(show\_regre\_result=True),则基于当前theta绘制回归曲线:
\begin{itemize}
   \item 生成X轴预测区间(x\_pred),其他特征固定为均值;
   \item 构建扩展特征矩阵(含偏置项),计算预测值y\_pred;
   \item 绘制红色回归直线,并添加图例(显示回归方程$y = \theta_0 + \theta_i x$);
\end{itemize}
\par  设置坐标轴标签、标题与网格线,通过canvas.draw()刷新绘图。

\subsection{关键技术特点}
\par  \textbf{实时交互性}:通过QTimer实现100ms级的训练过程可视化,用户可直观观察参数更新对回归曲线的影响;
\par  \textbf{状态一致性}:所有操作(如特征切换、算法变更)均通过修改类属性并调用update\_plot()保证界面与数据状态一致;
\par  \textbf{功能模块化}:预处理、训练、绘图等功能封装为独立方法,便于维护与扩展(如新增回归算法只需扩展regression\_choices与regression\_functions)。

该模块通过将PyQt5的交互框架与Matplotlib的可视化能力结合,为回归模型的训练过程提供了直观、可操作的界面支持,降低了算法调试与结果分析的门槛。